\documentclass[a4paper,10pt]{article}
\usepackage[cm]{fullpage}

\usepackage[top=1cm, bottom=1cm, left=1cm, right=1cm]{geometry}%A Few Useful Packages
\usepackage{marvosym}
\usepackage{fontspec}					%for loading fonts
 
\usepackage{xunicode,xltxtra,url,parskip} 	%other packages for formatting
\RequirePackage{color,graphicx}
\usepackage[usenames,dvipsnames]{xcolor}
\usepackage{fullpage} 				%better formatting of the A4 page
% an alternative to Layaureo can be ** \usepackage{fullpage} **
\usepackage{supertabular} 				%for Grades
\usepackage{titlesec}					%custom \section
\usepackage{multirow}

\usepackage{array}
%Setup hyperref package, and colours for links
\usepackage{hyperref}


\marginparsep = 0pt
\marginparwidth = 2pt

\definecolor{linkcolour}{rgb}{0,0.46,0.83}

\hypersetup{pdfauthor=Sylvain Blot,
			pdfsubject=Resume,
			colorlinks,breaklinks,urlcolor=linkcolour,
			linkcolor=linkcolour}

%FONTS
\defaultfontfeatures{Mapping=tex-text}
\setmainfont[SmallCapsFont = Fontin SmallCaps]{Fontin}


%CV Sections inspired by: 
%http://stefano.italians.nl/archives/26
\titleformat{\section}{\Large\scshape\raggedright}{}{0em}{}[\color{linkcolour}\titlerule]
\titlespacing{\section}{0pt}{3pt}{3pt}

\titleformat{\subsection}{\Large\raggedright}{}{0em}{}[]
\titlespacing{\subsection}{0pt}{3pt}{3pt}


%Tweak a bit the top margin
%\addtolength{\voffset}{-1.3cm}

%Italian hyphenation for the word: ''corporations''
\hyphenation{im-pre-se}

%-------------WATERMARK TEST [**not part of a CV**]---------------
%\usepackage[absolute]{textpos}

%\setlength{\TPHorizModule}{30mm}
%\setlength{\TPVertModule}{\TPHorizModule}
%\textblockorigin{2mm}{0.65\paperheight}
%\setlength{\parindent}{0pt}

%--------------------BEGIN DOCUMENT----------------------
\begin{document}

\pagestyle{empty} % non-numbered pages

%--------------------TITLE-------------
%\begin{tabular}[t]{p{7cm}p{10cm}}
%\Huge \textsc{Sylvain Blot} \\ \huge \textsc{Ingénieur Système} &  \normalsize \textsc{146 boulevard du Montarpasse - 75014 Paris} \\ \Large{\Letter}\large{\href{mailto:sylvain.blot@gmail.com}{sylvain.blot@gmail.com}} \\ \Large{\Telefon} \large{+ 33 6 75 86 87 19}\\ \normalsize \textsc{12 Février 1985 - Permis B} \\
%\end{tabular}


\begin{tabular}[T]{p{5cm}p{9,12cm}p{3cm}}
 \LARGE{\textsc{Sylvain Blot}}\newline \Large{\textsc{Chef de projet\newline Développement\newline \& Système}} & \normalsize{\textsc{6 rue du Jeu de Paume - 67000 Strasbourg}} \newline \Large{\Letter} \large{\href{mailto:sylvainblot@me.com}{sylvainblot@me.com}} \Large{\Telefon} \large{+ 33 6 75 86 87 19} \newline \normalsize{\textsc{12 Février 1985 - Permis de conduire}}
\newline \normalsize{\textsc{Français, Anglais, Allemand (bases)}} &\parbox[c]{1em}{\includegraphics[width=0.20\textwidth,angle=0]{sylvain.jpg} }
\end{tabular}
%--------------------SECTIONS-----------------------------------
%Section: Work Experience at the top
\section{Experiences professionelles}
%\begin{tabular}{r|p{11cm}}
\begin{tabular}{p{3,5cm}|p{14,5cm}}	

%FreeLance
	\emph{Septembre 2013} & \textsc{Chef de projet web, Smile Industries}, Strasbourg \\\textsc{Septembre 2012}&\emph{Développement d'un ERP, CRM, eCommerce. Gestion d'une équipe de 5 développeurs}\\&\footnotesize{Sécurisation de l'existant, amélioriation des performances, versioning, intégration continue, déploiements.}\\\multicolumn{2}{c}{} \\

%FreeLance
	\emph{Septembre 2012} & \textsc{Freelance Gestion de projet developement}, Strasbourg \\\textsc{July 2011}&\emph{Graphiques boursiers des marchés de matières premières, solution de réservation hotelière}\\&\footnotesize{Django, Drupal, Expression Engine.}\\\multicolumn{2}{c}{} \\

\emph{Juin 2011} & Chef de projet développement web \textsc{SdV Plurimédia}, Strasbourg \\\textsc{Avril 2010}&\emph{Création d'une solution de vente de voyages en ligne.}\\&\footnotesize{Symfony, Drupal, équipe de 4 personnes\hyperlink{sdv}{\hfill | \footnotesize Détails}}\\\multicolumn{2}{c}{} \\
  
%Mallyance
 \emph{Mars 2010} & Ingénieur système et réseaux chez \textsc{Mallyance}, Paris \\\textsc{Septembre 2008}&\emph{Chef de projet technique pour France Télécom - GOA}\\&\footnotesize{Gestion de l'infrastructure du jeu Warhammer Online.\hyperlink{mallyance}{\hfill | \footnotesize Détails}}\\\multicolumn{2}{c}{} \\
%Prestige
 \emph{Septembre 2007} & Première expérience chez \textsc{Prestige Réseaux}, Suresnes \\\textsc{Décembre 2006}&\emph{Consultant et formateur Linux}\\&\footnotesize{Formation de partenaires \emph{Novell} à Linux, missions de consulting chez grands comptes.\newline \hyperlink{prestige}{\hfill | \footnotesize Détails}}\\\multicolumn{2}{c}{} \\
%Supinfo SCT
 \emph{Juin 2007} & Formateur certifié \textsc{SUPINFO}, Strasbourg \\\textsc{Septembre 2005}&\emph{Formateur en technologies Linux et Sun Solaris}\\&\footnotesize{Formations des élèves d'\emph{ESI Supinfo} (jusqu'à BAC+5) au cours de l'année, présentation, pédagogie, examen, soutenances de projets.\newline \hyperlink{supinfo}{\hfill | \footnotesize Détails}}\\\multicolumn{2}{c}{}
%\textsc{Summer 2007} & Summer Intern at \textsc{Lehman Brothers}, \emph{Capital Markets}\\&\footnotesize{Received pre-placed offer from the Exotics Trading Desk as a result of very positive review. Rated ``\emph{truly distinctive}'' for Analytical Skills and Teamwork.}
\end{tabular}
%Section: Scholarships and additional info
%\section{Scholarships and Certificates}
%\begin{tabular}{rl}
% \textsc{Sept.} 2006 & Scholarship for graduate students with an outstanding curriculum \footnotesize(\EURcr 30,000)\normalsize\\
%\textsc{June} 2006 & {\textsc{Gmat}\textregistered}\setmainfont[SmallCapsFont=Fontin SmallCaps]{Fontin-Regular}: 730 (\textsc{q:50;v:39}) 96\textsuperscript{th} percentile; \textsc{awa}: 6.0/6.0 (89\textsuperscript{th} percentile)
%\end{tabular}
%Section: Languages
\section{Compétences et intérêts}
\begin{tabular}{p{3,5cm}|p{14,5cm}}	
\textsc{OS} & Mac OS X, Linux, Unix (Solaris, *BSD), Windows\\
\textsc{Langages} & C, JAVA, Python (PyGTK, wxWidget, QT), Objective C, Shell script (BaSH, Perl), SQL,\LaTeX\\
\textsc{Langages web} & HTML5, Python Django, PHP5(Symfony, Expression Engine, Drupal), Javascript, CSS\\
\textsc{Outils} & Versioning (Git, SVN, CVS), OpenSSH, DNS (BIND), LDAP, NFS, OpenSSL, SAMBA, IPTABLES, Kerberos, MySQL, Apache, Fabric, VMWare, VirtualBox\\
\textsc{Intérêts} & Literature, Raspberry PI hack, Puzzle\\
\textsc{Sports} & Course à pieds, natation, tennis, football
\end{tabular}


%Section: Education
\section{Formation}
%\begin{tabular}{rl}	
%\begin{tabular}{r|p{11cm}}
\begin{tabular}{p{3,5cm}|p{14,5cm}}	
 \textsc{Septembre} 2008 & Master of Science in \textsc{Computer Science} avec \textsc{Félicitations}, \textbf{Oxford Brookes University}, Oxford\\
 \multirow{3}*{\includegraphics[width=0.15\textwidth]{brookes.pdf}} & Membre de la British Computer Society\\
& Mémoire: ``Finger tracking Desktop Experience'' | \small Tuteur: Docteur Philip \textsc{Torr}\\
& Recherches sur le remplacement de la souris par la détection des doigts\\
& Domaines: systèmes distribués, IA, compilateur, production de logiciels \hyperlink{oxford}{\hfill | \footnotesize Liste des compétences}\\&\\\multicolumn{2}{c}{} \\
 \textsc{Septembre} 2008 & Titre d'ingénieur en \textsc{Informatique}, \textbf{ESI Supinfo}, Strasbourg\\
& Expert en systèmes d'information \\
 & OS : Linux, Microsoft, Apple; Réseaux; Base de données Oracle; Développement \\\multicolumn{2}{c}{} 
\end{tabular}

%\newpage
%\hypertarget{gmat}{\textsc{Gmat}\setmainfont{LMRoman10 Regular}\textregistered\setmainfont[SmallCapsFont=Fontin-SmallCaps]{Fontin-Regular}}

%\XeTeXpdffile ''GMAT.pdf'' page 1 scaled 800

\newpage
\section{Experiences professionelles}
\subsection{Chef de projet \textsc{SdV Plurimédia}}

\begin{itemize}
	\item \hypertarget{sdv}Création de backoffice "User friendly"
	\begin{itemize}
		\item Framework PHP Symfony
		\item Interface dynamique Javascript Jquery
		\item Parser XML haute performance
		\item Webservice XML-RPC
	\end{itemize}
	\item Création de sites de vente de voyages avec Drupal 6 et 7
	\begin{itemize}
		\item Alimenté par le backoffice via webservice
		\item Dévelopement d'un module de vente temp-réel par webservice
	\end{itemize}
\end{itemize}

\bigskip
\subsection{Chef de projet technique \textsc{GOA, Orange}}

\begin{itemize}
	\item \hypertarget{mallyance}Gestion transverse d’une équipe de 6 administrateurs réseaux basée en Irlande
	\begin{itemize}
		\item Création et attribution des tâches
		\item Administration de 750 serveurs, 60 univers de jeu
		\item Documentations
		\item Formation
		\item Communication uniquement en anglais.
	\end{itemize}
	\item Gestion fonctionnelle du jeu
	\begin{itemize}
		\item Ordonnancement des tâches
		\item Planification et organisation des maintenances du jeu
		\item Coordination des équipes (QA, Community, SysAdmin)
		\item Compilation, publication de clients de jeu Windows et Mac
	\end{itemize}
	\item Projets secondaires :
	\begin{itemize}
		\item Intégration application iPhone
		\item Sortie du jeu sous Mac OS
		\begin{itemize}
			\item Expérience utilisateur
			\item Création d’installeurs Mac
		\end{itemize}
	\end{itemize}
	\item Technologies :
	\begin{itemize}
		\item Patcheur haute disponibilité, loadbalancing
		\item Base de données MySQL
		\item Tomcat, Webservices Java
		\item Scripting: python, bash, dsh, AppleScript, Objective-C
	\end{itemize}
\end{itemize}

\bigskip
\subsection{Consultant et Formateur \textsc{Prestige Réseaux}}

\begin{itemize}
\item \hypertarget{prestige}Formation
	\begin{itemize}
	\item Formation des nouveaux partenaires Novell de l’année 2007 (environ 50 experts)
	\item Enseignement des cursus Novell Certified Linux Professional and Engineer
	\item Obtention des certifications Linux: Certified Linux Professional (cours 3071, 3072, 3073), Certified Linux Engineer (cours 3074, 3075)
 	\end{itemize}
\item Missions de consulting
	\begin{itemize}
	\item Virtualisation: Etude comparative de performances entre les solutions Xen et VMWare pour le groupe Lactalis
	\item Intégration : Spécifications techniques de serveurs SuSE Linux Enterprise Server 10 pour le centre décisionnel informatique de La Poste
		\begin{itemize}
		\item Configuration du système
		\item Sécurité
		\item Tolérance de panne sur architecture SAN (carte HBA, fiber channel)
		\item Création de repositories RPM personnalisés
		\end{itemize}
	\item Participation au déploiement de l’infrastructure mail de l’Elysée lors de la dernière prise de pouvoir
 	\end{itemize}
\item Avant-Vente : Présentation et mise en situation de la distribution SuSE Linux Enterprise Desktop 10 auprès d’Hermès
 
\item Veille technologique
	\begin{itemize}
	\item Rédaction de documentations techniques pour l’équipe de consultants
	\item Etude de mise en production du service d’envoi et de réception de fax Hylafax
	\item Démarrage réseau de systèmes Linux avec PXE par clients légers
	\item Annuaire eDirectory
	\end{itemize}
\end{itemize}

\bigskip

\subsection{Formateur UNIX/Linux \textsc{Supinfo}}

\begin{itemize}
	\item \hypertarget{supinfo}Membre du laboratoire des technologies Linux (www.labo-linux.com)
	\begin{itemize}
		\item  Responsable de l’équipe recherche
		\item  Pilotage de 20 personnes pendant 6 mois
		\item  Stage de 4 mois pour apprendre les techniques de formation, Supinfo Cerfied Trainer
		\item  Rédaction de supports de cours
		\item  Méthodologie de création de présentation PowerPoint
		\item  Prise de parole et animation de classe
	\end{itemize}
	\item Enseignement des technologies Linux et Solaris
	\begin{itemize}
		\item  Elèves de classe préparatoire jusqu’à Bac+4
		\item  Linux
		\begin{itemize}
			\item  Base de Linux, prise en main du système
			\item  Administration système et réseaux
			\item  DHCP, DND, LDAP, IPTABLES, APACHE, MySQL
		\end{itemize}
		\item  Solaris
		\begin{itemize}
			\item  FHS, disques, paquets, BootROM, SMF, Zones, RBAC, NFS
		\end{itemize}
		\item  Création de nombreux supports de cours, travaux pratiques, présentations et questions d’examens
	\end{itemize}
	\begin{itemize}
		\item Création de deux certifications professionnelles pour Mandriva : 
		\begin{itemize}
			\item Mandriva Certified User
			\item Mandriva Certified Administrator
			\item Comprennant support de cours, présentation, TP et examen
			\item Equipe de formateurs répartie dans le monde
			\item Contrôle qualité
		\end{itemize}
		\end{itemize}
\end{itemize}

\bigskip

\subsection{Master in Science in \textsc{Computer Science}}
\begin{itemize}
	\item \hypertarget{oxford}Mémoire : Finger Tracking Desktop Experience
		\begin{itemize}
			\item Recherche sur le remplacement de la souris par la détection des doigts
			\item Etude comparative des solutions de traitement d’images
			\item Choix porté sur l’usage d’une caméra infrarouge, du langage Python et du système d’exploitation Linux et son gestionnaire de fenêtre Gnome.
			\item Réalisation d’un prototype 100\% fonctionnel permettant :
			\begin{itemize}
				\item Avec 2 mains d’utiliser le poste linux
					\begin{itemize}
						\item Cliquer, double cliquer, drag and drop…
					\end{itemize}
				\item Reconnaissance des mouvements par réseaux de neurones
					\begin{itemize}
						\item Déclencher des raccourcis clavier
						\item Lancer des programmes
						\item Envoyer des messages au système via le bus de message
					\end{itemize}
				\item Interface Graphique en GTK
					\begin{itemize}
						\item Configuration des mouvements
						\item Attacher la caméra au système
					\end{itemize}
			\end{itemize}
			\item Mémoire écrit entièrement en Anglais
		\end{itemize}
 	\item Matières :
		\begin{itemize}
			\item Production de logiciels / Project Management
				\begin{itemize}
					\item Spécifications techniques
					\item Cycles de développent
					\item Suivie de projet
					\item Nombreux cas pratiques étudiés en Java
				\end{itemize}
				\item Systèmes Distribués
				\begin{itemize}
					\item Architecture, Sécurité, Virtualisation
					\item P2P, Corba, MPI, PVM, DFS, SOAP
					\item Tolérance de panne, GRID computing, Intelligence artificielle
					\item Réalisation de systèmes distribués en langage C
				\end{itemize}
			\item Création de compilateur
				\begin{itemize}
					\item Lexer/Parser en Java
					\item Interpretation µJava
				\end{itemize}
			\item Intelligence Artificielle
				\begin{itemize}
					\item Systèmes experts
					\item Data Mining
					\item Agents d'interface
				\end{itemize}
		\end{itemize}
\end{itemize}

\end{document}
